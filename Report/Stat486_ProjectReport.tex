\documentclass{article}
\usepackage{graphicx}
\usepackage{fancyhdr,extramarks}
\usepackage{enumitem}
\usepackage{amsmath}
\usepackage{amssymb}
\usepackage{xcolor, soul}
\usepackage{blindtext}
\usepackage{titlesec}
\usepackage[titles]{tocloft}
\usepackage[titles]{tocloft}
\usepackage{url}
\usepackage{hyperref}

% Formatting
\topmargin=-0.40in
\evensidemargin=0in
\oddsidemargin=0in
\textwidth=6.5in
\textheight=9.0in
\headsep=0.25in
\linespread{1.1}
\setlength\parindent{24pt}

% Page layout
\pagestyle{fancy}
\lhead{\Title}
\chead{\Course}
\rhead{\Topic}
\cfoot{\thepage}
\renewcommand\headrulewidth{0.4pt}
\renewcommand\footrulewidth{0.0pt}

\newcommand{\Title}{Final Project Report}
\newcommand{\Course}{{\bf 01:960:486}}
\newcommand{\Topic}{Nilay, Katharina, HaoYang, Rajeev}

\begin{document}

\section*{Introductory Description}
\hspace{\parindent} 

    The data set \href{https://archive.ics.uci.edu/dataset/2/adult}{adult} from UC Irvine's Machine Learning Repository in 1996 was extracted from the 1994 Census database, with a total of 15 variables to predict whether an individual's annual income exceeds $\$50$k

    The 15 variables were respectively: \texttt{age, workclass, fnlwgt, education, educationnum, $\linebreak$ martialstatus, occupation, relationship, race, sex, capitalgain, capitalloss, hoursperweek, nativecountry, 50k}

    First, we removed variables \texttt{capitalgain, capitalloss} simply due to most observations having missing values for them, then we removed variables \texttt{fnlwgt} as we note \href{https://www.kaggle.com/datasets/uciml/adult-census-income}{here}, \texttt{fnlwgt} being heavily determined by \texttt{age, race, sex} will produce high multicollinearity. We removed \texttt{education} for the same reason, as it's simply categorical representations of \texttt{educationnum}. We then removed \texttt{relationship} as it also causes multicollinearity with \texttt{sex}, as husband/wife would be associated with male/female. We lastly removed \texttt{nativecountry} simply because there are too many different categorical responses, with some holding so few observations while the variable itself has many missing values. 

     For further data cleaning, we first re-coded \texttt{sex, 50k} to binary variables, and \textbf{Federal-gov, Local-gov, State-gov} under \texttt{workclass} all to \textbf{gov}. We then added $3$ to every observation in \texttt{educationnum}, the $\#$ of years an individual spent in education, as it was strangely counting 3 years less (for example 10th grade in \texttt{education} only having \texttt{educationnum} of 7). Lastly, we removed the observations that have missing values in \texttt{workclass, occupation}, further the 21 observations with responses \textbf{Never-worked} and \textbf{Without-pay}, the dataset still has 30703 observations after data cleaning.

\section*{Exploratory Data Analysis}
    \subsection*{Outlier Detection}
    \hspace{\parindent} 

    
    \subsection*{Summary Statistics}
    \hspace{\parindent} 

    
    \subsection*{Data Visualization}
    \hspace{\parindent} 

    
\section*{Prediction Algorithms/Results}
    \subsection*{Support Vector Machines (SVN)}
    \hspace{\parindent} 


    \subsection*{Random Forest}
    \hspace{\parindent} 

    
\section*{Qualitative Discussion}
\hspace{\parindent} 

    
\end{document}
