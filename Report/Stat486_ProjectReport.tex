\documentclass{article}
\usepackage{graphicx}
\usepackage{fancyhdr,extramarks}
\usepackage{enumitem}
\usepackage{amsmath}
\usepackage{amssymb}
\usepackage{xcolor, soul}
\usepackage{blindtext}
\usepackage{titlesec}
\usepackage[titles]{tocloft}
\usepackage[titles]{tocloft}
\usepackage{url}
\usepackage{hyperref}

% Formatting
\topmargin=-0.40in
\evensidemargin=0in
\oddsidemargin=0in
\textwidth=6.5in
\textheight=9.0in
\headsep=0.25in
\linespread{1.1}
\setlength\parindent{24pt}

% Page layout
\pagestyle{fancy}
\lhead{\Title}
\chead{\Course}
\rhead{\Topic}
\cfoot{\thepage}
\renewcommand\headrulewidth{0.4pt}
\renewcommand\footrulewidth{0.0pt}

\newcommand{\Title}{Final Project Report}
\newcommand{\Course}{{\bf 01:960:486}}
\newcommand{\Topic}{Nilay, Katharina, HaoYang, Rajeev}

\begin{document}

\section*{Introductory Description/Data Cleaning}
\hspace{\parindent} 

    The data set \href{https://archive.ics.uci.edu/dataset/2/adult}{adult} from UC Irvine's Machine Learning Repository in 1996 was extracted from the 1994 Census database, with a total of 15 variables to predict whether an individual's annual income exceeds $\$50$k

    The 15 variables were respectively: \texttt{age, workclass, fnlwgt, education, educationnum, $\linebreak$ martialstatus, occupation, relationship, race, sex, capitalgain, capitalloss, hoursperweek, nativecountry, 50k}

    First, we removed variables \texttt{capitalgain, capitalloss} simply due to most observations having missing values for them, then we removed variables \texttt{fnlwgt} as we note \href{https://www.kaggle.com/datasets/uciml/adult-census-income}{here}, \texttt{fnlwgt} being heavily determined by \texttt{age} and \texttt{sex} will produce high multicollinearity. We removed \texttt{education} for the same reason, as it's simply categorical representations of \texttt{educationnum}. We then removed \texttt{relationship} as it also causes multicollinearity with \texttt{sex}, as husband/wife would be associated with male/female. We removed \texttt{race} as its categories were too broad for us to expect significant findings. We lastly removed \texttt{nativecountry} simply because there are too many different categorical responses, with some holding so few observations while the variable itself has many missing values. 

     For further data cleaning, we first re-coded \texttt{sex, 50k} to binary variables, and \textbf{Federal-gov, Local-gov, State-gov} under \texttt{workclass} all to \textbf{gov}. We then added $3$ to every observation in \texttt{educationnum}, the $\#$ of years an individual spent in education, as it was strangely counting 3 years less (for example 10th grade in \texttt{education} only having \texttt{educationnum} of 7). Lastly, we removed the observations that have missing values in \texttt{workclass, occupation}, further the 21 observations with responses \textbf{Never-worked} and \textbf{Without-pay}, the dataset still has 30703 observations after data cleaning.

\section*{Exploratory Data Analysis}
    \subsection*{Outlier Detection}
    \hspace{\parindent} 

    As we are predicting a binary variable, we can't easily find outliers numerical for the response variable, so we look at the 3 numerical features, \texttt{age}, \texttt{educationnum}, and \texttt{hoursperweek}. As we have more than 30000 observations from the census, we may confidently use standardized methods, we will also use the iqr test.
    
    The range for \texttt{age} is 17 to 90. First, we standardize the distribution, and we find 122 observations with absolute values of $z$-scores larger than 3 with their ages ranging 78 to 90. Then, we find 172 observations that fall outside of the range of $(q1-1.5(\text{iqr}), q3+1.5(\text{iqr}))$ with the range being 76 to 90. The median and mean methods granting similar results would indicate \texttt{age} is not heavily skewed (verified by $\mu \approx 38.4$, $q2 = 37$), and should be treated as a normal distribution. A potential outlier subset of $<200$ observations should also not influence the data heavily.

    The range for \texttt{educationnum} is 4 to 19, with 19 being the standard number of years of education for someone with a doctorate degree. We find the exact same 202 observations that have 4 or 5 years of education to be the outliers under both the standardized $z$ test and the iqr test, thus no heavy skewness expected, which is again verified by $\mu \approx 13.1$ while $q2 = 13$. We also don't expect these outliers to be impactful to the total data due to its small size.

    The range for \texttt{hoursperweek} is 1 to 99, We find 450 observations that take values in $[1, 4] \cup [77, 99]$ through the standardized $z$ test, but through the iqr test, we find a total of 8092 observations that take value in 74 out of the original data's 94 unique values to be potential outliers. Although we wouldn't expect 450 potential outliers to heavily impact the data, 8092 is concerning. We find that $\mu \approx 41$ while $q2 = 40$, thus we may temporarily assign the cause of this being the distribution of \texttt{hoursperweek} having high kurtosis, and reasonably only treat the original 450 observations as potential outliers for \texttt{hoursperweek}.

    \subsection*{Summary Statistics}
    \hspace{\parindent} 

    
    \subsection*{Data Visualization}
    \hspace{\parindent} 

    
\section*{Prediction Algorithms/Results}
    \subsection*{Support Vector Machines (SVN)}
    \hspace{\parindent} 


    \subsection*{Random Forest}
    \hspace{\parindent} 

    
\section*{Qualitative Discussion}
\hspace{\parindent} 

    
\end{document}
